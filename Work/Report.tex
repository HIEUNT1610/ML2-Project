\documentclass[12pt]{article}

\usepackage{graphicx}
\usepackage{fullpage}
\usepackage{natbib}
\usepackage{amsmath}


\title{Retrofitting Word Embeddings using Semantic Lexicons}
\author{Ngo Trung Hieu, Zheng Peng, Xu Sun \thanks{M1 Linguistique Informatique - Université Paris Cité}}
\bibliographystyle{plainnat}
\date{}

\begin{document}
\maketitle

\section{Introduction}
`is it possible to use the semantic lexicons to improve the quality of word embeddings?'
\section{Requirements}
Report+code submission
Upload your report and code on moodle, as a single archive, which should include clearly separated

the program(s), with online help and a README allowing simple use,
and a report including about 20 pages is requested.
The report must contain:
a theoretical part , which describes in a well-separated manner: (i) the task to be automated, (ii) the method to perform this task automatically, its advantages and disadvantages;
an experiments and results section: here you describe the resources that you used (corpora, lexicon etc...), the different experiments performed, the method of evaluation of your system, and the results of this evaluation. 
a computer science part , which very briefly describes the general organization of the code, possibly implementation choices, possible problems encountered and their solution, etc. ;
a « user manual » : the practical indications (command line) to execute the program. The best way is to program an online help, then copy and paste the online help into your report.
NB: even if the jury knows the task, you must re-explain it in your own words.

NB: if you use articles, scientific books and/or resources (lexicons, annotated corpora...), cite them specifically (including refs provided by your supervisor).
\section{Phases of the project}

\section{Modelisation}

\section{Encountered difficulties and non-finished tasks}

\section{Conclusion}

\end{document}